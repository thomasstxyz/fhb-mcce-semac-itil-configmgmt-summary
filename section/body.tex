\section{ITIL Service configuration management}

\subsection{Zweck}
Der Zweck der Servicekonfigurationsmanagement-Praktik ist es, sicherzustellen,
dass genaue und zuverlässige Informationen über die Konfiguration von
Diensten und die sie unterstützenden Konfigurationsobjekte zur Verfügung
stehen, wann und wo sie benötigt werden. Dazu gehören Informationen darüber,
wie Konfigurationselemente konfiguriert sind und welche
Beziehungen zwischen ihnen bestehen. 

Organisationen nutzen Ressourcen, um
Produkte und Dienste zu erstellen und zu liefern. Diese Ressourcen können der
Organisation gehören, oder sie können als Teil eines Dienstes verfügbar
sein, den die Organisation konsumiert. Das Servicekonfigurationsmanagement
sammelt und verwaltet Informationen über eine Vielzahl von Ressourcen, wie 
z. B. Hardware, Software, Netzwerke, Gebäude, Mitarbeiter, Lieferanten, Produkte,
Dienstleistungen und Dokumentation. Die Ressourcen, die in den
Anwendungsbereich der Praktik fallen, werden als Konfigurationselemente (CIs)
bezeichnet.

\begin{center}
	\textbf{Konfigurationselement} (engl. configuration item (CI)):
	Jede Komponente, die verwaltet werden muss, um einen IT-Dienst zu erbringen.
\end{center}

\noindent
Das Hauptziel der Servicekonfigurationsmanagement-Praktik ist die effiziente
Bereitstellung nützlicher Informationen für die Organisation.
Der Umfang der Komponenten, die unter der Kontrolle stehen, sollte durch
ihre Nützlichkeit und Effizienz definiert werden. Die wichtigsten Faktoren,
die diese Praktik prägen, sind die Nützlichkeit der
Konfigurationsinformationen und die Kosten für ihre Beschaffung und Pflege.

Ressourcen, die nicht einzeln verwaltet werden können, gelten in der Regel
nicht als CIs. Zum Beispiel das Wissen, das von einem Analysten zur Verwaltung
von Vorfällen genutzt wird, ist wichtig, wird aber wahrscheinlich nicht als CI
behandelt. 

Die Praktik des
Servicekonfigurationsmanagements konzentriert sich auf Ressourcen, die wichtig
für Produkt- und Servicemanagement sind, unabhängig davon, ob
diese Ressourcen Eigentum der Organisation oder als Teil 
eines Drittanbieter-Dienstes bereitgestellt werden. Für
diese beiden Ressourcengruppen können unterschiedliche Lebenszyklus-Modelle und
Kontrollen gelten.

Die Praktik des Servicekonfigurationsmanagements beinhaltet die Identifizierung
und Dokumentation der Verbindungen und Beziehungen zwischen
Konfigurationselementen. Dies führt in der Regel zu Modellen
(bekannt als Servicekonfigurationsmodelle, Service-Resourcing-Modelle oder
funktionale und finanzielle Service Modelle). Diese Modelle können sich auf
verschiedene Aspekte der Servicearchitektur und die Beziehungen zwischen den
Komponenten beziehen. 

Servicekonfigurationsmodelle werden auf verschiedene Weise verwendet, 
unter anderem für:

\begin{itemize}
    \item Auswirkungsanalyse
    \item Ursachen- und Wirkungsanalyse
    \item Risikoanalyse
    \item Kostenzuweisung
    \item Verfügbarkeitsanalyse und -planung.
\end{itemize}

\noindent
Servicekonfigurationsmodelle sollten so gestaltet und gepflegt werden, dass sie
den Bedürfnissen der Beteiligten entsprechen. 
Das Servicekonfigurationsmanagement ist ein hoch automatisiertes Verfahren. Es
stützt sich auf die Sammlung, Pflege und Kontrolle großer Mengen von
Konfigurationsdaten und umfasst häufig Erstellung, Pflege und Präsentation
komplexer Konfigurationsmodelle. Die Praktik umfasst Daten aus verschiedenen
Quellen zu sammeln, zu integrieren und auf sinnvolle Weise zu präsentieren.
In der Regel werden spezialisierte Tools zusammen mit Überwachungs-,
Erkennungs-, Analyse- und Aufzeichnungssystemen verwendet.

Diese Praktik ermöglicht oft die Integration von Konfigurationsinformationen
mit Aufzeichnungen, die als Teil anderer Praktiken verwaltet werden, einschließlich Incident Management, Change Enablement, Problem Management, Monitoring und 
Event Management sowie Service Request Management, unter anderem. 
Einige Konfigurationsmodelle sind jedoch
schwer oder gar nicht zu automatisieren, sie erfordern manuelle Datenpflege
und/oder Beziehungszuordnung. Beispiele hierfür sind Benutzerdaten,
Organisationsstrukturen, Verträge mit Lieferanten und Partnern usw. Manueller
Aufwand und die damit verbundenen Kosten sollten neben den
Automatisierungs- und Integrationskosten berücksichtigt werden, wenn die
Praktik entworfen und verbessert wird.

\subsection{Ergebnisse der Praktik}
Wie jede andere ITIL-Management-Praktik trägt auch die Servicekonfigurationsmanagement-Praktik
einen Beitrag zu mehreren Wertströmen. Es ist wichtig, daran zu denken, dass ein Wertstrom niemals
für eine einzelne Praktik gebildet wird. Die Servicekonfigurationsmanagement-Praktik wird kombiniert mit anderen
Praktiken, um Kunden qualitativ hochwertige Dienstleistungen zu bieten. Die wichtigsten Aktivitäten der Wertschöpfungskette, zu denen
die Praktik beiträgt, sind:

\begin{itemize}
	\item deliver and support
	\item design and transition
	\item obtain/build
	\item improve.
\end{itemize}

\noindent
Für die ausgewählten Arten von CIs stellt das Servicekonfigurationsmanagement sicher, dass:

\begin{itemize}
	\item vertrauenswürdige Konfigurationsdaten bereitgestellt und gepflegt werden, was die Aktualisierung der Dienste und der Konfigurationsdaten beinhaltet, 
	um laufende Änderungen des Status, der Attribute und der Beziehungen von CIs zu berücksichtigen.
	\item relevante und genaue Berichte zur Unterstützung der Entscheidungsfindung bereitgestellt werden
	\item der CI-Lebenszyklus in andere Verfahren integriert ist.
\end{itemize}

\noindent
Es gibt mehrere Aktivitäten und Verantwortungsbereiche, die nicht in die Praktik 
des Servicekonfigurationsmanagement gehören, obwohl sie eng mit dem Servicekonfigurationsmanagement verbunden sind.
In einigen Fällen hängt die Praktik von diesen Aktivitäten ab. 

\subsection{Wichtige Begriffe und Abläufe}
\textbf{Konfigurationsmanagementsystem} (engl. Configuration management system):
Eine Reihe von Werkzeugen, Daten und Informationen, die zur Unterstützung des Servicekonfigurationsmanagements verwendet werden
Praxis. \\

\noindent
\textbf{Konfigurationsmanagementdatenbank} (engl. Configuration management database
(CMS, CMDB)):
Eine Datenbank, in der Konfigurationsdaten während ihres gesamten Lebenszyklus gespeichert werden. Sie verwaltet auch die
Beziehungen zwischen Konfigurationsdatensätzen. \\

\noindent
\textbf{Basis-Konfiguration} 
(engl. Baseline configuration):
Eine Konfiguration eines Produkts, einer Dienstleistung oder einer Infrastruktur, die formell überprüft und
vereinbart wurde. Sie dient als Grundlage für weitere Aktivitäten, wie Nutzung, Entwicklung und Planung. \\

\noindent
\textbf{Verifizierung} 
(engl. Verification):
Eine Aktivität, die sicherstellt, dass ein neuer oder geänderter IT-Dienst, Prozess, Plan oder eine andere Leistung
mit seiner Entwurfsspezifikation übereinstimmt und vollständig, genau und zuverlässig ist. \\

\noindent
\textbf{CMDB Audit} 
(engl. CMDB audit):
Eine geplante, strukturierte und dokumentierte Inspektion der Konfigurationselemente der Organisation, die
die darauf abzielt, die Korrektheit der CMDB-Daten im Geltungsbereich zu bewerten. \\

\noindent
\textbf{Prozess} 
(engl. Process):
Eine Reihe von miteinander verbundenen oder interagierenden Aktivitäten, die Inputs in Outputs umwandeln. Ein Prozess nimmt
einen oder mehrere definierte Inputs und wandelt sie in definierte Outputs um. Prozesse definieren die Abfolge von
Aktionen und ihre Abhängigkeiten. \\

\noindent
Die Aktivitäten des Servicekonfigurationsmanagements umfassen drei Prozesse:

\subsection*{Verwaltung eines gemeinsamen Ansatzes für das Dienstkonfigurationsmanagement}

Dieser Prozess zielt darauf ab, einen effektiven und effizienten Ansatz für das Management von
Konfigurationsinformationen in der Organisation zu etablieren. 
Zu den wichtigsten Aktivitäten, die der Prozess umfasst, zählen:

\begin{itemize}
	\item Analyse der Stakeholder Requirements
	\item Definieren des Ansatzes des Servicekonfigurationsmanagements
	\item Kommunikation des Servicekonfigurationsmanagements-Ansatz in der Organisation
	\item Review und Anpassung des Servicekonfigurationsmanagements-Ansatz und Prozeduren
\end{itemize}

\subsection*{Erfassen, Verwalten und Bereitstellen von Konfigurationsinformationen}

Dieser Prozess konzentriert sich auf die Aktualisierung, Wartung und Bereitstellung von Konfigurationsinformationen.
Zu den wichtigsten Aktivitäten, die der Prozess umfasst, zählen:

\begin{itemize}
	\item Analyse der Ressourcen und Identifikation von CIs
	\item Bestätigen des CI Modells
	\item Folgen des CI Modells
	\item Abhandeln von Ausnahmen
	\item Review des CI Modells
\end{itemize}

\subsection*{Überprüfen von Konfigurationsdaten}

Dieser Prozess ist darauf ausgerichtet, dass die Konfigurationsdaten vollständig, korrekt und konform sind.
Zu den wichtigsten Aktivitäten, die der Prozess umfasst, zählen:

\begin{itemize}
	\item Identifikation des CI Modells
	\item Verifizierung von Konfiugrationsdaten
	\item Review des Verifizierungsoutputs
	\item Definition und Implementierung von Korrekturmaßnahmen
	\item Verfassen und Übermitteln eines CMDB-Verifizierungsbericht
\end{itemize}

\subsection{Practice Success Factors (Erfolgsfaktoren)}
Ein Practice Success Factor (PSF) - zu Deutsch Erfolgsfaktor der Praktik -
ist mehr als schlicht eine simple Tätigkeit. Ein PSF beinhaltet Komponenten
von allen vier Dimensionen von Servicemanagement, die gemeinsam sicherstellen,
dass die Praktik effektiv ist. Die Praktik des Servicekonfigurationsmanagements umfasst die folgenden PSFs:

\begin{itemize}
	\item Sicherstellen, dass die Organisation über relevante Konfigurationsinformationen über ihre Produkte und Dienstleistungen verfügt
\end{itemize}

\noindent
Der Schwerpunkt der Servicekonfigurationsmanagement-Praktik liegt auf der Sicherstellung, dass relevante Konfigurationsinformationen
erfasst, gepflegt und den Stakeholdern bei Bedarf zur Verfügung gestellt werden.
Typischerweise nutzen die Stakeholder, die von den Konfigurationsinformationen profitieren, diese im Zusammenhang mit anderen
Managementpraktiken und im weiteren Kontext der Wertströme der Organisation.
Die Konfigurationsinformationen sollten für die Bedürfnisse des Unternehmens relevant sein. Alle verfügbaren
Daten in eine CMDB aufzunehmen oder den Beispielen anderer Organisationen blind zu folgen, ist nicht sinnvoll. Das
Servicekonfigurationsmanagement ist nur so wertvoll wie die Informationen, die es liefert
genau, aktuell, zuverlässig, verständlich, einfach zu verwenden und relevant sind.

\begin{itemize}
	\item Sicherstellen, dass die Kosten für die Bereitstellung von Konfigurationsinformationen kontinuierlich optimiert werden
\end{itemize}

\noindent
Das Servicekonfigurationsmanagement liefert Informationen für andere Verfahren und fungiert als
eine unterstützende Praktik in den meisten Wertströmen der Organisation. Es ist jedoch unwahrscheinlich, dass sie
zu einem Wertstrom mit zentralen wertschöpfenden Aktivitäten beitragen wird. Das bedeutet, dass es wichtig ist
dass die Kosten für Konfigurationsinformationen kontinuierlich optimiert werden müssen.

\section{Example usage in context}
